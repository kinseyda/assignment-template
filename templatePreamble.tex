\usepackage[utf8]{inputenc}
\usepackage[margin=25mm]{geometry}
\usepackage[T1]{fontenc}
\usepackage{enumitem}
\usepackage{amsmath}
\usepackage{amssymb}
\usepackage{amsthm}
\usepackage{hyperref}
\usepackage{bm}
\usepackage{graphicx}
\usepackage{lastpage}
\usepackage{fancyhdr}
\usepackage{accents}
\usepackage{pgfkeys}
\usepackage[usenames,dvipsnames]{xcolor}
\usepackage{tcolorbox}
\usepackage{cancel}
\usepackage{algorithm}
\usepackage[noend]{algpseudocode}
\usepackage{pgffor}
\usepackage{xcolor}
\usepackage{minted}
\usepackage{placeins}
\usepackage{adjustbox}
\usepackage{mathtools}
\usepackage{tikz}
\usetikzlibrary{automata, arrows, positioning, fit}

\tikzset{node distance = 3cm, on grid, auto} % good setup for DFAs and such

\pagestyle{fancy}

\definecolor{LightGray}{gray}{0.9}
\setminted{breaklines,frame=lines,framesep=2mm,baselinestretch=1.2,bgcolor=LightGray,fontsize=\footnotesize,linenos,tabsize=4}


\geometry{
	left=25mm,
	right=25mm,
	headheight=25mm,
	top=35mm,
	bottom=25mm,
	footskip=10mm
}

\hypersetup{
	colorlinks = true,
	linkcolor = blue
}

% Set and get arbitrary variables using pgfkeys
% Usage:
% "\sv{x = \frac{3}{4}}" ... "\gv{x}"
\newcommand{\sv}[1]{\pgfkeys{/variables/#1}}
\newcommand{\gv}[1]{\pgfkeysvalueof{/variables/#1}}
\newcommand{\declare}[1]{%
	\pgfkeys{
		/variables/#1.is family,
		/variables/#1.unknown/.style = {\pgfkeyscurrentpath/\pgfkeyscurrentname/.initial = ##1}
	}%
}

% Creates a bright red box with the text inside, useful for TODO notes to ensure
% you dont forget to do something before submitting the assignment.
\NewDocumentCommand{\todo}{m}{
	\begin{tcolorbox}[colback=red!5!white,colframe=red!75!black,title=TODO]
		#1
	\end{tcolorbox}
}


\declare{}

\delimitershortfall=-1pt % for automatically sizing brackets

\newcommand{\lbr}{\left\lparen}
\newcommand{\rbr}{\right\rparen}
\newcommand{\lsb}{\left\lbrack}
\newcommand{\rsb}{\right\rbrack}
\newcommand{\lcb}{\left\lbrace}
\newcommand{\rcb}{\right\rbrace}
% \br{} makes a bracket / parenthesis pair "()" For example: \br{x + y} will
% produce (x + y), and \br{1 + \br{2 * 4}} will produce (1 + (2 * 4)), but with
% nice, expanding sizes of brackets. You should use these instead of \left( and
% \right) directly, it works better with IDEs too.
\newcommand{\br}[1]{{\lbr #1 \rbr}}
% \sbr{} makes a square bracket pair "[]"
\newcommand{\sbr}[1]{{\lsb #1 \rsb}}
% \cbr{} makes a curly brace pair "{}"
\newcommand{\cbr}[1]{{\lcb #1 \rcb}}


% 
\makeatletter
\renewenvironment{proof}[1][\proofname]{\par
	\pushQED{\qed}%
	\normalfont \topsep6\p@\@plus6\p@\relax
	\trivlist
	\item\relax
	{\itshape
		#1\@addpunct{.}\/}
	\hspace\labelsep\ignorespaces
}
{
	\popQED\endtrivlist\@endpefalse
}

% Use this environment for solutions, it will put a "QED" symbol at the end and format it nicely.
\newenvironment{solution}
{\renewcommand\qedsymbol{$\blacksquare$}
	\begin{proof}[Solution]}
		{\end{proof}}
\renewcommand\qedsymbol{$\blacksquare$}

% Use this environment for question parts that are brackets around letters (a), (b), (c), ...
\newenvironment{alpQ}{\begin{enumerate}[label= (\alph*)]}
		{\end{enumerate}}

% A horizontal separator line 
\NewDocumentCommand{\midsep}{}{
	\begin{center}
		\rule{0.5\textwidth}{.4pt}
	\end{center}
}

% For vertically splitting stuff
% Params are: content left, content right, title left, title right
\NewDocumentCommand{\lsplit}{m m O{} O{}}{
	\begin{minipage}[t]{.45\textwidth}
		\begin{center}
			\textbf{#3}
		\end{center}
		#1
	\end{minipage}%
	\hspace{0.025\textwidth}
	\vrule
	\hspace{0.025\textwidth}
	\begin{minipage}[t]{.45\textwidth}
		\begin{center}
			\textbf{#4}
		\end{center}
		#2
	\end{minipage}
}

\allowdisplaybreaks % for pagebreaks in mathmode