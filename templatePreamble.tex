\usepackage[utf8]{inputenc}
\usepackage[margin=25mm]{geometry}
\usepackage[T1]{fontenc}
\usepackage{enumitem}
\usepackage{amsmath}
\usepackage{amssymb}
\usepackage{amsthm}
\usepackage{hyperref}
\usepackage{bm}
\usepackage{listings}
\usepackage{graphicx}
\usepackage{lastpage}
\usepackage{fancyhdr}
\usepackage{accents}
\usepackage{pgfkeys}
\usepackage[usenames,dvipsnames]{xcolor}
\usepackage{tcolorbox}
\usepackage{cancel}
\usepackage{algorithm}
\usepackage[noend]{algpseudocode}
\usepackage{pgffor}
\pagestyle{fancy}

\geometry{
	left=25mm,
	right=25mm,
	headheight=25mm,
	top=35mm,
	bottom=25mm,
	footskip=10mm
}

\hypersetup{
	colorlinks = true,
	linkcolor = blue
}

% eg "\sv{x = \frac{3}{4}}" ... "\gv{x}"
\newcommand{\sv}[1]{\pgfkeys{/variables/#1}}
\newcommand{\gv}[1]{\pgfkeysvalueof{/variables/#1}}
\newcommand{\declare}[1]{%
	\pgfkeys{
		/variables/#1.is family,
		/variables/#1.unknown/.style = {\pgfkeyscurrentpath/\pgfkeyscurrentname/.initial = ##1}
	}%
}

\declare{}

\delimitershortfall=-1pt % for automatically sizing brackets

\lstdefinestyle{R}{
	language=R,                     % the language of the code
	basicstyle=\footnotesize\ttfamily, % the size of the fonts that are used for the code
	numbers=left,                   % where to put the line-numbers
	numberstyle=\tiny\color{Blue},  % the style that is used for the line-numbers
	stepnumber=1,                   % the step between two line-numbers. If it is 1, each line will be numbered
	numbersep=5pt,                  % how far the line-numbers are from the code
	backgroundcolor=\color{white},  % choose the background color. You must add \usepackage{color}
	showspaces=false,               % show spaces adding particular underscores
	showstringspaces=false,         % underline spaces within strings
	showtabs=false,                 % show tabs within strings adding particular underscores
	frame=single,                   % adds a frame around the code
	rulecolor=\color{black},        % if not set, the frame-color may be changed on line-breaks within not-black text (e.g. commens (green here))
	tabsize=2,                      % sets default tabsize to 2 spaces
	captionpos=b,                   % sets the caption-position to bottom
	breaklines=true,                % sets automatic line breaking
	breakatwhitespace=false,        % sets if automatic breaks should only happen at whitespace
	keywordstyle=\color{RoyalBlue},      % keyword style
	commentstyle=\color{YellowGreen},   % comment style
	stringstyle=\color{ForestGreen}      % string literal style
}                                                                                                                                                                     % string literal style% comment style% keyword style% sets if automatic breaks should only happen at whitespace% sets automatic line breaking% sets the caption-position to bottom% sets default tabsize to 2 spaces% if not set, the frame-color may be changed on line-breaks within not-black text (e.g. commens (green here))% adds a frame around the code% show tabs within strings adding particular underscores% underline spaces within strings% show spaces adding particular underscores% choose the background color. You must add \usepackage{color}% how far the line-numbers are from the code

\lstdefinestyle{mono}{
	basicstyle=\footnotesize\ttfamily,
	breaklines=true
}

\lstdefinestyle{Python}{
	language=Python,
	basicstyle=\ttm,  basicstyle=\footnotesize,        % size of fonts used for the code
	breaklines=true,                 % automatic line breaking only at whitespace
	captionpos=b,                    % sets the caption-position to bottom
	commentstyle=\color{YellowGreen},    % comment style
	keywordstyle=\color{RoyalBlue},       % keyword style
	stringstyle=\color{ForestGreen},     % string literal style
	frame=single,                   % adds a frame around the code
	numbers=left,                   % where to put the line-numbers
	numberstyle=\tiny\color{Blue},  % the style that is used for the line-numbers
	stepnumber=1,                   % the step between two line-numbers. If it is 1, each line will be numbered
	numbersep=5pt, % how far the line-numbers are from the code
}

\makeatletter
\renewenvironment{proof}[1][\proofname]{\par
	\pushQED{\qed}%
	\normalfont \topsep6\p@\@plus6\p@\relax
	\trivlist
	\item\relax
	{\itshape
		#1\@addpunct{.}\/}
	\hspace\labelsep\ignorespaces
}
{
	\popQED\endtrivlist\@endpefalse
}

\newenvironment{solution}
{\renewcommand\qedsymbol{$\blacksquare$}
	\begin{proof}[Solution]}
		{\end{proof}}
\renewcommand\qedsymbol{$\blacksquare$}

\newenvironment{alpQ}{\begin{enumerate}[label= (\alph*)]}
		{\end{enumerate}}

\NewDocumentCommand{\midsep}{}{
	\begin{center}
		\rule{0.5\textwidth}{.4pt}
	\end{center}
}

% For vertically splitting stuff
% content left, content right, title left, title right
\NewDocumentCommand{\lsplit}{m m O{} O{}}{
	\begin{minipage}[t]{.45\textwidth}
		\begin{center}
			\textbf{#3}
		\end{center}
		#1
	\end{minipage}%
	\hspace{0.025\textwidth}
	\vrule
	\hspace{0.025\textwidth}
	\begin{minipage}[t]{.45\textwidth}
		\begin{center}
			\textbf{#4}
		\end{center}
		#2
	\end{minipage}
}

\allowdisplaybreaks % for pagebreaks in mathmode